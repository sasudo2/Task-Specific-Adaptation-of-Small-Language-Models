\chapter{FEASIBILITY ANALYSIS}

This project is primarily targeted at fine-tuning large language models for Python code generation using Parameter Efficient Fine-Tuning (PEFT) with QLoRA. The feasibility of this project is analyzed from the following perspectives:

\section*{Operational Feasibility}
The project is designed to be applicable on relatively low-end devices. The hardware requirement for Llama-2 7B is approximately 6-8GB of VRAM, which is widely available in consumer-grade GPUs such as the NVIDIA RTX 3060 or 4060, as well as the Google Colab free tier (T4 GPU). 



The use of QLoRA significantly reduces the memory footprint, making it feasible to execute fine-tuning on a single consumer-grade GPU rather than requiring an enterprise-level server cluster.

\section*{Economic Feasibility}
The project utilizes freely available open-source models (Llama-2) and public datasets (FlyTec, StaQC). Since the hardware constraints are low enough to utilize existing consumer-grade GPUs or free cloud-based services like Google Colab, the primary development cost is zero. If extended training time is required beyond the free tier limits, Google Colab Pro can be utilized at a nominal cost of approximately \$10 per month.

\section*{Technical Feasibility}
The implementation relies on well-established, industry-standard libraries such as Hugging Face \texttt{transformers}, \texttt{peft}, and \texttt{bitsandbytes}. These libraries provide robust support for 4-bit quantization and LoRA adapter integration. The team possesses the necessary proficiency in Python programming and machine learning frameworks to manage the model loading, data preprocessing, and evaluation pipelines effectively.